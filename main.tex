\documentclass{article}
\usepackage[utf8]{inputenc}

\title{Analýza použití androidu}
\author{}

\usepackage{booktabs}
\usepackage{placeins}

\begin{document}
\maketitle
\pagenumbering{gobble}
\newpage
%============================ vse k analyze je potreba pridat \usepackage tak jak je vyse
\section*{Analýza použití androidu}
\paragraph{} Výběr podporovaného androidu je založen na procentuelním zastoupení všech zařízení používajících tento systém. Dle dat získaných z oficiálních stránek je momentální procentuelní zastoupení následující:	(data přečtena k 10.10. 2018)

\begin{table}[h]
    \centering
    \renewcommand{\tablename}{Tabulka}
    \begin{tabular}{c|l|c}
        \toprule
        \textbf{Verze} & \textbf{Název} & \textbf{Zastoupení}\\
        \midrule
        4.4 & KitKat & 7.8\% \\
        5.0 & Lollipop & 3.6\% \\
        5.1 &  & 14.7\% \\
        6.0 & Marshmallow & 21.6\% \\
        7.0 & Nougat & 19.0\% \\
        7.1 &  & 10.3\% \\
        8.0 & Oreo & 13.4\% \\
        8.1 &  & 5.8\% \\
        \bottomrule
    \end{tabular}
    \caption{Procentuální zastoupení androidu}
    \label{Tabulka}
\end{table}

\paragraph{} Tabulka výše pokrývá 96.2 \% všech zařízení. Toto procentuální pokrytí je více než dostačující, jak toto číslo vypovídá nebude problém nasadit aplikaci na drtivé většině zařízení, která jsou stále v oběhu. Starší verze androidu nebudou podporovány, hlavní důvod je výše zmíněné pokrytí, dalším důvodem je možný nedostatek starší verze systému. Ten může postrádat některé důležité funkcionality, které budou potřeba pro některé hry, případně celou aplikaci. Jako příklad uvedeme podporu animací, špatné vykreslování 2D grafiky, možná nestabilita starších verzí systému, ... .

\paragraph{} Ze všech informací, které byli získány pomocí oficiálních stránek androidu, případně ověřena na těchto stránkách byl vybrán android KitKat (verze 4.4) jako poslední podporovaný systém. 

%====================================================== konec textu

\end{document}